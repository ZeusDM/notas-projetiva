\chapter{Definindo o Plano Projetivo}

\section{Plano Euclidiano}

Você talvez esteja acostumado com a noção do Plano Euclidiano.
Essa estrutura envolve muitos conceitos, incluindo (mas não limitado a):
\begin{enumerate}[label = \textbullet]
	\item pontos,
	\item retas,
	\item segmentos de reta,
	\item distância,
	\item ângulos,
	\item polígonos (triângulos, quadriláteros, \dots),
	\item circunferências,
	%\item conics (ellipses, parabolas, and hyperbolas),
	\item transformações (rotações, translações, reflexões, \dots).
\end{enumerate}

Apesar de todas esses conceitos estarem empacotados no Plano Euclidiano, parece que os conceitos mais fundamentais são os primeiros dessa lista.
Em outras palavras, os conceitos de \emph{pontos} e \emph{retas} parecem ser mais simples e mais relacionados à definição de Plano Euclidiano, do que os conceitos de \emph{ângulos} e \emph{transformações}.

Nós poderíamos gastar algumas folhas formalizando a definição de Plano Euclidiano (usando o que os matemáticos chamam de \emph{axiomas}) e nos converncer de que, de fato, \emph{pontos} e \emph{retas} são intimamente relacionados ao que realmente o Plano Euclidiano é.
Porém, nós prefirimos não fazer isso.
Na verdade, vamos listar duas propriedades que o Plano Euclidiano \emph{parece} ter:
\begin{enumerate}[label = (\textit{\roman*})]
	\item Para cada par de pontos distindos, existe uma única reta passando por ambos. \label{item:pe:i}
	\item Para cada par de retas distintas, existe um único ponto que está em ambas. \label{item:pe:ii}
\end{enumerate}

\begin{exer} \label{exer:propriedadesdoplanoeuclidiano}
	Descubra se as propriedades acima são verdades ou não no Plano Euclideano.
\end{exer}

\section{Plano Projetivo}

No Exercício \ref{exer:propriedadesdoplanoeuclidiano}, você talvez tenha notado que, embora a propriedade \ref{item:pe:i} seja verdadeira, \ref{item:pe:ii} não é.
Existe até um nome especial para um par de retas que não se encontra em nenhum ponto: dizemos que essas retas são \emph{retas paralelas}.

% Aside: We could here define a relation between pair of lines, namely "line \(\ell_1\) is parallel to line \(\ell_2\)",
% and deduce that this relation is an equivalence relation; thus generates equivalence classes.

%In fact, one of the axioms of the Euclidean Plane directly contradicts \ref{item:pe:ii}:
%\begin{quote} % www.math.brown.edu/tbanchof/Beyond3d/chapter9/section01.html 
%	For every given point not on a given line, there exists a unique line passing through the given point that does not meet the given line.
%\end{quote}

Mas... as propriedades \ref{item:pe:i} e \ref{item:pe:ii} são tão bonitas (pelo menos pra nós).
É tão triste que elas não são verdadeiras.
Bem, nós somos matemáticos!
Nós podemos fazer o que quisermos com nossas definições (apesar de ter que viver com suas consequências).

\begin{pdefn}
	Um \emph{plano projetivo} é uma estrutura que consiste em três coisas:
	\begin{enumerate}[label = \textbullet]
		\item um conjunto \(\mathcal P\) de pontos,
		\item um conjunto \(\mathcal L\) de retas, e
		\item uma noção de incidência, i.e., se \(P\) é um ponto e \(\ell\) é uma reta, nós poderemos dizer se \(\ell\) passa (ou não) pelo ponto \(P\).
	\end{enumerate}

	Adicionalmente, precisamos das seguintes propriedades:
	\begin{enumerate}[label = (\textit{\roman*})]
		\item Para cada par de pontos distindos, existe uma única reta passando por ambos. \label{item:pp:i}
		\item Para cada par de retas distintas, existe um único ponto que está em ambas. \label{item:pp:ii}
	\end{enumerate}
\end{pdefn}
