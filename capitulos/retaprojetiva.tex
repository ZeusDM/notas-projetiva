\chapter{Reta Projetiva}

\section{Definição}

\begin{defn}[Reta Projetiva]
A \emph{reta projetiva real \(\mathbb{P}^1\mathbb{R}\)} é o conjunto das retas do \(\mathbb{R}^2\) que passam pela origem.
Elementos da reta projetiva real são chamados de \emph{pontos projetivos}.
\end{defn}

Como um abuso de notação, chamaremos a reta projetiva real somente de reta projetiva.
Quando o significado estiver claro, podemos chamar a reta projetiva de reta, e o ponto projetivo de ponto.

Em termos de Álgebra Linear, os pontos da reta projetiva são os subespaços do $\mathbb{R}^2$ com dimensão~$1$.

Podemos inferir uma representação dos pontos projetivos da reta projetiva a partir dos vetores do \(\mathbb{R}^2\).
Se \(\mathbf{v} \neq (0, 0)\), vamos chamar de \([\mathbf{v}]\) a única reta que passa pela origem e por \(\mathbf{v}\).
Lembre-se que uma reta no \(\mathbb{R}^2\) é definida por dois pontos distintos; portanto, a reta \([\mathbf{v}]\) está bem definida.

De modo mais preciso, podemos concluir que \([\mathbf{v}]\) é o conjunto dos múltiplos reais de \(\mathbf{v}\), isto é, é o conjunto \(\{ \lambda \mathbf{v} : \lambda \in \mathbb{R}\}\).

Porém, essa representação não é única.
Por exemplo, \([(1, 2)]\), \([(2, 4)]\) e \([(-1, -2)]\) representam o mesmo ponto projetivo, isto é, a mesma reta.
De modo geral, se \(\lambda \neq 0\), os pontos projetivos \([\mathbf{v}]\) e \([\lambda \mathbf{v}]\) são os mesmos.

Apesar da representação não ser única, temos que \([\mathbf{v}] = [\mathbf{u}]\) se, e somente se, \(\mathbf{u}\) é um múltiplo real de \(\mathbf{u}\).
Em outras palavras, \([\mathbf{v}] = [\mathbf{u}]\) se, e somente se, existe \(\lambda \in \mathbb{R}\) tal que \(\mathbf{u} = \lambda\mathbf{v}\).

Portanto, o sistema de coordenadas da reta projetiva é um pouco diferente do sistema que estamos acostumados, mas continua sendo útil mesmo assim. Chamamos sistemas como esse, em que múltiplicar por um escalar não muda o ponto, de \emph{sistema de coordenadas homogêneas}.

\begin{exmp}
Os pontos projetivos $[1,0]$, $[0,1]$, $[1,1]$ e $[2,1]$ correspondem respectivamente ao eixo $x$, ao eixo $y$, à reta $y=x$ e à reta $y=\frac{1}{2}x$ do $\mathbb{R}^2$.
\end{exmp}

Mas por que chamamos a reta projetiva de ``reta''? Uma maneira de responder essa pergunta é, dada uma base do $\mathbb{R}^2$, considerar a reta $y=1$. Com isso, podemos tomar como representante para cada ponto projetivo o vetor que leva da origem até sua interseção  com a reta $y=1$.

(inserir imagem)

Contudo, vemos que exatamente um dos pontos projetivos, a reta \(y = 0\), não intersecta a reta \(y = 1\).
Para essa reta, vamos escolher o representante $[(1,0)]$.
Portanto, \[\mathbb{P}^1\mathbb{R} = \{ [(x,1)] : x \in \mathbb{R} \} \cup \{[(1,0)]\},\]
ou seja, podemos interpretar reta projetiva real como uma reta real com um ponto extra.

\section{Projetividades}
