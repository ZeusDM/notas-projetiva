\chapter{Reta Projetiva}

Inicialmente definiremos o que são os pontos projetivos a partir do $\mathbb{R}^2$:

\begin{defn}[Ponto Projetivo]
Vamos definir uma relação de equivalência \(\sim\) entre vetores do $\mathbb{R}^2 \setminus \{(0, 0)\}$,
onde dizemos que \((a,b) \sim (c,d)\) se, e somente se, existe real não nulo \(\lambda\) tal que \[(a,b)=\lambda (c,d).\]

Denotaremos então um ponto projetivo na reta por sua classe de equivalência $[a,b]$ induzida pelo ponto euclidiano $(a,b)$.
\end{defn} 

Em termos de algebra linear, cada uma dessas classes de equivalência é um subespaço de dimensão $1$ do $\mathbb{R}^2$.

Assim, estamos fazendo uma correspondência entre um ponto projetivo e uma reta que passa pela origem do $\mathbb{R}^2$.

Note que, em termos de coordenadas, só estamos interessados na razão entre as coordenadas de um ponto euclidiano (exceto quando uma delas é zero, o que é fácil de lidar).

\begin{exmp}
De acordo com a nossa definição, os pontos projetivos $[1,0]$, $[0,1]$, $[1,1]$ e $[2,1]$ correspondem respectivamente ao eixo $x$, ao eixo $y$, à reta $y=x$ e à reta $y=\frac{1}{2}x$ do plano euclidiano.

(inserir desenho)
\end{exmp}


\begin{defn}[Reta Projetiva]
Definimos a reta projetiva como o conjunto de todos os pontos projetivos como definidos acima e a denotaremos por $\mathbb{P}(\mathbb{R})$.

\end{defn}

Mas por que chamamos esse conjunto de "reta"? Uma maneira de responder essa pergunta é tomando uma reta $r$ arbitrária (que não passa pela origem) do plano euclidiano e considerando as suas interseções com o feixe de retas que passa pela origem. Isso nos faz escolher um representante para cada ponto projetivo, com exceção do ponto correspondente à reta paralela a $r$, que denotaremos por ponto do infinito. 

(inserir desenho)

Dessa forma, se tomarmos $r$ como sendo a reta de equação $y=1$, podemos escrever

$$\mathbb{P}(\mathbb{R})=\{[x,1]\mid x\in \mathbb{R}\}\cup {[1,0]}$$

Veja que essa definição de ponto do infinito é compatível àquela dada na primeira parte deste livro.
