\chapter{Reta Projetiva}

\section{Definição}

\begin{defn}[Reta Projetiva]
Vamos chamar de reta projetiva o conjuntos de todas as retas que passam pela origem do $\RR^2$. E chamaremos de pontos projetivos cada uma dessas retas.
\end{defn}
Usualmente denotamos esse conjunto por $\PP(\RR)$.

Em termos de álgebra linear, cada um dos elementos de $\PP(\RR)$ é um subespaço de dimensão $1$ do $\RR^2$.

Duas possíveis perguntas que surgem é : 
\begin{enumerate}
    \item Por que chamamos isso
do mesmo nome da "reta projetiva"  definida anteriormente? %(inserir referencia pra definição de reta projetiva no capítulo sem alglin)
    \item Se a gente consegue sistema de coordenadas pro $\RR^2$, será que também conseguimos algo parecido para o $\PP(\RR)$?
\end{enumerate}

Começando pela segunda, se temos um sistema de coordenadas pro $\RR^2$ -- ou seja, uma base -- termos que cada vetor $(a,b) \ne (0,0)$ estaria em um único ponto projetivo, contudo cada ponto projetivo possui uma família de vetores que pertencem a ele. Como os pontos projetivos são as retas que passam pela origem, se $(a,b)$ pertencem a um ponto projetivo $P$ então o conjunto de elementos de $P$ são os vetores da forma $\alpha \cdot (a,b), \ \alpha \in \RR$.

Com isso podemos tomar um sistema de coordenadas pro $\PP(\RR)$, mas esse sistema de coordenas é homogêneo, ou seja, a representação de cada ponto não é única, pois $(a,b)$ e $\alpha \cdot (a,b)$ representam o mesmo ponto. Vamos denotar as coordenadas de pontos projetivos entre colchetes, 

\begin{exmp}
De acordo com a nossa definição, os pontos projetivos $[1,0]$, $[0,1]$, $[1,1]$ e $[2,1]$ correspondem respectivamente ao eixo $x$, ao eixo $y$, à reta $y=x$ e à reta $y=\frac{1}{2}x$ do $\RR^2$.
\end{exmp}

Voltando na primeira pergunta. Mas por que chamamos esse conjunto de "reta"? Uma maneira de responder essa pergunta é, dada uma base do $\RR^2$, considerar a reta $y=1$. Com isso, podemos tomar como representante para cada ponto projetivo o vetor que leva da origem até sua interseção  com a reta $y=1$.

(inserir imagem)

Contudo, vemos que exatamente um dos pontos projetivos não tem imagem nessa reta. Para ele vamos escolher o representante $[1,0]$, ou seja:

$$\PP(\RR)=\{[x,1]\mid x\in \RR\}\cup {[1,0]}$$


Com isso essa definição é compatível com aquela dada na primeira parte deste livro de uma reta com um ponto extra.

\section{Projetividades}