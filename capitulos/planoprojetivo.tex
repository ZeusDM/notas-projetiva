\chapter{Plano Projetivo}

\section{Definição}

\begin{defn}[Plano Projetivo Real] \label{defn:planoprojetivoreal}
O \emph{plano projetivo real \(\mathbb{P}^2(\mathbb{R})\)} é o conjunto das retas do \(\mathbb{R}^3\) que passam pela origem.
Elementos do plano projetivo real são chamados de \emph{pontos projetivos}.
\end{defn}

Essa definição é completamente análoga à Definição \ref{defn:retaprojetivareal}; apenas aumentamos uma dimensão.

Como um abuso de notação, chamaremos o plano projetivo real somente de plano projetivo.
Quando o significado estiver claro, podemos chamar o plano projetivo de plano.

Em termos de Álgebra Linear, os pontos do plano projetivo são os subespaços do \(\mathbb{R}^3\) com dimensão~\(1\).

Podemos inferir uma representação dos pontos projetivos do plano projetivo a partir dos vetores do \(\mathbb{R}^3\), de modo completamente análogo ao que fizemos na Seção \ref{sec:defnretaprojetiva}.

Se \(\mathbf{v} \neq (0, 0, 0)\), vamos chamar de \([\mathbf{v}]\) a única reta que passa pela origem e por \(\mathbf{v}\); equivalentemente, também podemos definir \([\mathbf{v}]\) como \(\langle v \rangle\), o subespaço do \(\mathbb{R}^3\) gerado por \(\mathbf{v}\).
Podemos concluir que \([\mathbf{v}]\) é o conjunto dos múltiplos reais de \(\mathbf{v}\), isto é, é o conjunto \(\{ \lambda \mathbf{v} : \lambda \in \mathbb{R}\}\).

Portanto, temos novamente que \([\mathbf{v}] = [\mathbf{u}]\) se, e somente se, \(\mathbf{u}\) é um múltiplo real de \(\mathbf{v}\).
Em outras palavras, \([\mathbf{v}] = [\mathbf{u}]\) se, e somente se, existe \(\lambda \in \mathbb{R}\) tal que \(\mathbf{u} = \lambda\mathbf{v}\).

Por que chamamos o plano projetivo de ``plano''? Uma maneira de responder essa pergunta é, dada uma base do $\mathbb{R}^3$, considerar o plano $z=1$.
Com isso, podemos tomar como representante para cada ponto projetivo o vetor que leva da origem até sua interseção com o plano $z=1$.

Contudo, vemos que vários pontos projetivos não intersectam o plano \(z = 1\).
Mais precisamente, as retas que passam pela origem e estão contidas no plano \(z = 0\) são os pontos projetivos que não possuem representante pela regra acima.
Porém, o conjunto de pontos projetivos contidos em um plano real possui nome!
Chamamos esses conjuntos de reta projetiva.

Portanto, podemos interpretar a reta projetiva real como um plano real com uma reta projetiva extra.
Juntando com o que desenvolvemos na Seção \ref{sec:defnretaprojetiva}, podemos dizer que a reta projetiva real é um plano real, mais uma reta real, mais um ponto extra.
Podemos ver isso claramente em \ref{eqn:planoprojetivoplanoretaponto}.
\begin{equation} \label{eqn:planoprojetivoplanoretaponto}
  \mathbb{P}^2(\mathbb{R}) = \{ [x, y, 1] : x, y \in \mathbb{R} \} \cup \{[x, 1, 0] : x \in \mathbb{R}\} \cup \{[1, 0, 0]\}.
\end{equation}

\begin{exer}
  Convença a si mesmo de que a igualdade de conjuntos em \ref{eqn:planoprojetivoplanoretaponto} é verdadeira.
\end{exer}

\begin{thm} \label{thm:quatropontos}
  Dados quatro pontos projetivos distintos \(A\), \(B\), \(C\) e \(D\) podemos escolher uma base apropriada do \(\mathbb{R}^3\) para a qual \(A = [1, 0, 0]\),  \(B = [0, 1, 0]\), \(C = [0, 0, 1]\) e \( D = [1, 1, 1]\)
\end{thm}
\begin{proof}
Prova análoga ao teorema \ref{thm:trespontos} 
\end{proof}
\section{Colinearidade e concorrência}
\begin{defn}
Dado um plano que passa pela origem do $\RR^3$, os conjunto de todas as retas contidas nele que passam pela origem, é chamado de \emph{reta projetiva}.
\end{defn}

Outra forma de pensar a reta projetiva contida no plano projetivo é ver ela como o espaço projetivo derivado de qualquer subespaço vetorial de dimensão dois do $\RR^3$.

\begin{thm}
Dada uma base para o $\RR^3$ tal que os pontos do plano projetivo $A$,$B$ e $C$ tem coordenadas $[x_A,y_A,z_A], [x_B,y_B,z_B]$ e $[x_C,y_C,z_C]$ respectivamente. 
\[A,B \text{ e } C \text{ são colineares (pertencem a uma mesma reta)} \iff \det \begin{bmatrix} x_A & y_A & z_A \\ x_B & y_B & z_B \\ x_C & y_C & z_C \end{bmatrix} = 0\]
\end{thm}

\begin{proof}
Se os pontos são colineares os representantes pertencem ao mesmo plano logo eles são \textit{linearmente dependentes}, com isso temos que o determinante é zero.

Já se o determinante é zero os representantes são \textit{linearmente dependente} logo pertencem ao mesmo plano.
\end{proof}

Isso mostra que dado $a$ e $b$ representantes fixos para $A$ e $B$ todos os pontos projetivos que são colineares com eles podem ser escritos da forma $\alpha a + \beta b$ com $\alpha,\beta \in \RR$. Ou seja, tomando $a$ e $b$, como "bases" dessa reta, podemos dizer que os pontos dessa reta tem coordenadas $[\alpha,\beta]$ na reta, ou seja coordenadas de uma reta projetiva, isso nos permite calcular razão cruzada de pontos colineares no plano projetivo.

Contudo, sabemos que um plano no $\RR^3$ pode ser descrito como, dada uma base, o conjunto de vetores $(x,y,z)$ tais que $ax+by+cz = 0$ com $a,b$ e $c$ reais fixos, mas se pegarmos um ponto projetivo qualquer, se um dos representantes dele pertence ao plano então todo o ponto também pertence, ou seja, temos que um ponto projetivo $P = [x,y,z]$ pertence a uma reta projetiva $r$ se, e somente se, vale que $ax+by+cz = 0$, sendo $a,b$ e $c$ relacionados a equação plano real associado $r$, com isso vamos dizer que as coordenadas da reta $r$ são $<a,b,c>$, note também que a coordenada das retas formam um sistema de coordenadas homogêneas pois $<a,b,c>$ e $<2a,2b,2c>$ representam o mesmo plano no $\RR^3$ logo a mesma reta no $\PP^2$

\begin{thm}[Primeiro Teorema de Desargues]
Dados dois triângulos $ABC$ e $A'B'C'$ as seguintes condições são equivalentes:
\begin{enumerate}
    \item $AA'$, $BB'$ e $CC'$ são concorrentes.
    \item $X = AB \cap A'B'$, $Y = AC\cap A'C'$ e $Z = BC \cap B'C'$ são colineares.
\end{enumerate}
\end{thm}

\begin{proof}
Vamos mostrar que $1 \implies 2$:

Seja $O = AA' \cap BB'$, pelo teorema \ref{thm:quatropontos} vamos tomar $A = [1, 0, 0]$, $B = [0, 1, 0]$, $C = [0, 0, 1]$ e $O = [1, 1, 1]$. Com isso temos que $AO = < 0, 1, -1>$, $BO = < 1, 0, -1>$ e $C) = < 1, -1, 0>$. Com isso temos que $A' = [a,1,1]$, $B' = [1,b,1]$ e $C'= [1,1,c]$, para alguns $a,b$ e $c$ reais. 

Agora vamos calcular $X$, $Y$ e $Z$. Note que: $AB = <0,0,1>$ e $A'B' = <b-1,a-1,1-ab>$, com isso $X = [a,-b,0]$, analogamente temos $Y = [-a,0,c]$ e $Z = [0,b,-c]$, como:

\[\det \begin{bmatrix} a & -b & 0 \\ -a & 0 & c \\ 0 & b & -c \end{bmatrix} = 0 \text{ temos que } X,Y \text{ e } Z \text{ são colineares } \]
\end{proof}


%parte para ser add no futuro
\begin{lem}[Lema Útil]
Se uma projetividade de uma $r$ em uma reta $s$ preserva um ponto então ela é uma projeção.
\end{lem}

\begin{thm}[Feixe de Projetividade]
Dada uma projetividade $\alpha$ de uma reta $r$ em outra $s$, seja $X = A\alpha(B) \cap B\alpha(A)$, temos que variando $A$ e $B$ em $r$, $X$ pertence a uma reta fixa, chamada \emph{feixe de $\alpha$}.
\end{thm}

\begin{cor}[Teorema de Pappus]
Dado $A$, $B$ e $C$ em uma reta e $A'$, $B'$ e $C'$ em outra, tome $X = AB' \cap A'B$, $Y = AC'\cap A'C$ e $Z = BC' \cap B'C$, temos que $X$, $Y$ e $Z$ são colineares.
\end{cor}


\begin{thm}[Segundo Teorema de Desargues]
Dado um quadrilátero $ABCD$ e uma reta fixa $r$, seja $X = AB \cap r$, $X' = CD \cap r$, $Y = AC \cap r$, $Y' = BD \cap r$, $Z = AD \cap r$ e $Z' = BC \cap r$, com isso temos que $(X,X')$, $(Y,Y')$ e $(Z,Z')$ são pares de uma involução na reta $r$.
    
\end{thm}