\chapter{Álgebra Linear}

Assumimos que o leitor esteja acostumado com conceitos relacionados à Álgebra Linear, especialmente em \(\mathbb{R}^2\) e \(\mathbb{R}^3\).
Definições e conceitos importantes incluem:
corpo, espaço vetorial, combinação linear, independência linear, espaços gerados, base de um espaço vetorial, dimensão de um espaço vetorial.
De qualquer modo, apresentaremos nesse capítulo um resumo desses conceitos.

\begin{defn}
    Falamos que o vetor \(\mathbf u\) é uma combinação linear entre os vetores \(\mathbf v_1\), \(\mathbf v_2\), \(\dots\), \(\mathbf v_n\), quando quando existem escalares \(\alpha_1\), \(\alpha_2\), \(\dots\), \(\alpha_n\) tais que
    \begin{equation}
        \mathbf u = \alpha_1\mathbf v_1 + \alpha_2\mathbf v_2 + \dots + \alpha_n\mathbf v_n.
    \end{equation}
\end{defn}

\begin{defn}
    Os vetores \(\mathbf v_1\), \(\mathbf v_2\), \(\dots\), \(\mathbf v_n\) são \emph{linearmente independentes} quando, se \(\alpha_1\), \(\alpha_2\), \(\dots\), \(\alpha_n\) são escalares satisfazendo
    \begin{equation}
        \alpha_1\mathbf v_1 + \alpha_2\mathbf v_2 +\dots + \alpha_n\mathbf v_n = 0,
    \end{equation}
    então
    \begin{equation}
        \alpha_1 = \alpha_2 = \dots = \alpha_n = 0.
    \end{equation}
\end{defn}

\begin{defn}
    O espaço gerado pelos vetores \(\mathbf v_1\), \(\mathbf v_2\), \(\dots\), \(\mathbf v_k\) é o conjunto de todas as possíveis combinações lineares entre eles, denotado por \(\langle\mathbf v_1,\mathbf v_2,\dots,\mathbf v_k\rangle\).
    Equivalentemente, é o menor subespaço vetorial contendo \(\mathbf v_1\), \(\mathbf v_2\), \(\dots\), \(\mathbf v_n\).
\end{defn}

\begin{defn}
    Se \(\mathbf v_1, \mathbf v_2, \dots, \mathbf v_n \in V\) são vetores linearmente independente tais que
    \begin{equation}
        \langle \mathbf v_1, \mathbf v_2, \dots, \mathbf v_n \rangle = V,
    \end{equation}
    então chamamos \(\{\mathbf v_1, \mathbf v_2, \dots, \mathbf v_n\}\) de \emph{base} de \(V\).
\end{defn}

\begin{thm}
    Dado um espaço vetorial \(V\), todas as suas bases possuem a mesma cardinalidade.
\end{thm}

\begin{defn}
    Chamamos de \emph{dimensão} de um espaço vetorial \(V\) o número de vetores em suas bases.
\end{defn}
