\chapter{Álgebra Linear}

\section{Algumas definições}

Antes de partimos para o tópico principal é necessário que algumas definições estejam claras.

%\subsection{Função}
\begin{defn}


$f$ é uma função do conjunto A no conjunto B (denota-se $f: \text{A} \to \text{B}$) se:
%\[ f = \{(a,b) \in A \times B | \ \forall a \in A \ \exists ! \ (a,b) \in f \}\]

\[f \subset A \times B \text{ tal que } \forall a \in A \ \exists ! \ b \in B \ | \ (a,b) \in f \]

Denotamos $f(a) = b \iff (a,b) \in f$
\end{defn}


\begin{defn}

%Uma operação binária \emph{estrela} em um conjunto X é, por definição, uma função de X $\times$ X em X, ou seja se denotarmos $x$ operação $x'$ como $x \star x'$ temos:

%\[ x \star x' =  y \iff f(x, x') = y\]


Uma operação binária \emph{"estrela"} em um conjunto A é, por definição, uma função de $A$ cartesiano $A$ em $A$, ou em matematiques:

\[ \star : A \times A \to A \]

Denotamos $a \star b = c \iff \star(a,b) = c$

Vale uma observação que pela nossa definição de função seria justo que estivesse escrito $\star((a,b))$, já que o elemento no qual aplicamos a função é o $(a,b)$, mas por uma certa comodidade ou talvez preguiça omitimos um dos parenteses. 

\end{defn}

\begin{defn}

Um corpo $\KK$ é um conjunto e duas operações, usualmente chamadas de adição e multiplicação, ou seja $\KK = \{C, +, \cdot\}$. Essas operações tem que ter algumas propriedades, sendo elas, para $a,b,c \in C$:

\begin{enumerate}
    \item Associativa:
    
        $a + (b + c) = (a + b) + c$
        $a \cdot (b \cdot c) = (a \cdot b) \cdot c = a \cdot b \cdot c$
        
    \item Comutativa:
    
    $a + b = b + a$
    
    $a \cdot b = b \cdot a$
    
    \item Elemento neutro:
    
    $\exists  e \in C \ | \ \forall a \in C, \ a + e = e + a = a$
    
    $\exists  e' \in C \ | \ \forall a \in C, \ a \cdot e' = e' \cdot a = a$
 
    \item Inverso:
    
   $\forall a \in C \ \exists \ a' \in C \ | \ a + a' = a' + a = e$
 
    $\forall a \ne e \in C \ \exists \ a' \in C \ | \ a \cdot a' = a' \cdot a= e'$
 
    \item Distributiva 
    
    $(a + b) \cdot c = a \cdot c + a \cdot b$
\end{enumerate}

É usual chamar o $e$ de $0$, o \textbf{$e'$} de $1$, o inverso aditivo de $a$ de $-a$ e o multiplicativo de $a^{-1}$ mesmo quando não estamos falando dos números reais. Além disso também é usual denotarmos $ a \in \KK $ ao invés de  $a \in C$.

\end{defn}

Dadas essa definições são bons exercícios mostrar que, para $a,b,c \in \KK $:
\begin{enumerate}
    \item $a \cdot (-1) = -a$
    \item $a \cdot 0 = 0$
\end{enumerate}



\section{Até que enfim Espaço}

\begin{defn}

Um espaço vetorial $V$ sobre o corpo $\KK$ é um conjunto munido de duas operações, $\oplus$ e $\odot$. Sendo $\oplus: V \times V \to V \text{ e } \odot: \KK \times V \to V$

%Um espaço vetorial é um conjunto, um corpo e duas operações, uma do espaço no espaço e uma operação do corpo no espaço, ou seja:


Sendo que estas possuem algumas propriedades, para $v,u,w \in V \text{ e } \alpha, \beta \in \KK$, temos:

\begin{enumerate}
    \item Associativa em $\oplus$ e $\odot$
    
        $v \oplus (u \oplus w) = (v \oplus u) \oplus w $
        
        $\alpha \odot (\beta \odot v) = \beta \odot (\alpha \odot v) = (\alpha \cdot \beta)\odot v$
    \item Comutativa em $\oplus$
    
        $v \oplus u = u \oplus v$
    \item Elemento neutro
    
     $\exists e \in V \ | \  \forall v \in V, v \oplus e = e \oplus v = v$, usualmente denotamos esse elemento como 0.
     
     $1 \odot v = v$, sendo 1 o elemento neutro da multiplicação em $\KK$.
     
     \item Inverso
     
     $\forall v \in V , \exists v' \in V \ | \ v \oplus v' = e$ usualmente denotamos $v'$ por $-v$ 
    \item Distributiva
    
    $(\alpha + \beta)\odot v = \alpha \odot v \oplus \beta \odot v$
    
    $\alpha \odot (v \oplus u)= \alpha \odot v + \alpha \odot u$
\end{enumerate}
\end{defn}

Vale observar, que como em corpos, também é usual simplificarmos um pouco a notação, usando o mesmo símbolo para soma (+) e o produto por escalar é usualmente denotado apenas pela justaposição dos elementos do corpo e do espaço ($\alpha v$), ainda no produto vale ressaltar que só está definido $\alpha v$, logo sempre que estiver escrito $v \alpha$, queremos dizer $\alpha v$.

Em resumo, tudo que der pra escrever menos, nós vamos escrever menos.

São bons exercícios mostrar que:

\begin{enumerate}
    \item $0v=0$
    \item $(-1)v = -v$
\end{enumerate}

\begin{defn}

Chamamos o espaço vetorial $V'$ de um subespaço de $V$ se: 
\begin{enumerate}
    \item $V'$ esta definido sobre o mesmo corpo.
    \item $\forall v \in V', v \in V$
    \item As operações de $V'$ são as operações de $V$ com o domínio e contradomínio restritos à $V'$. 
    
%    \item $\forall (v,u) \in V' \times V'$
%    
%    $v+u \in V'$
%    
%    $ -v \in V'$
%    \item $0 \in V'$
\end{enumerate}
\end{defn}

\subsection{Coisas relacionadas com base}

As  definições a seguir podem parecer um pouco desconexas separadas, contudo após elas todas estarem claras, o motivo da existência de cada uma fica mais claro também.


\begin{defn}
Falamos que o vetor $u$  é uma combinação linear entre os vetores $v_1,v_2,\allowbreak\dots,v_n$, quando quando existe $\alpha_1, \alpha_2, \cdots , \alpha_n \in \KK$ tal que $u = \alpha_1v_1 + \alpha_2v_2 +\dots + \alpha_nv_n $.
\end{defn}

\begin{defn}
Chamamos os vetores $v_1,v_2, \dots, v_n$ de linearmente independentes quando nenhum deles podem ser escritos como uma combinação linear dos outros vetores.
\end{defn}

\begin{cor}
$v_1,v_2, \dots, v_n$ são LI (linaermente independes) $\iff \alpha_1v_1 + \alpha_2v_2 +\dots + \alpha_nv_n = 0 \implies \alpha_1 = \alpha_2 = \dots = \alpha_n =0$.
\end{cor}

\begin{defn}
Um espaço $V'$ gerado pelos vetores $v_1,v_2,\dots,v_n$ de $V$ é o menor subespaço de $V$ que contém $v_1,v_2,\dots , v_n$.

Obs.: "Menor" significa que não existe nenhum subespaço que contenha esses vetores e  esteja contido em $V'$, tirando o próprio $V'$.
\end{defn}
    
\begin{lem}
O espaço gerados pelos vetores $v_1,v_2, \dots, v_n$ é o conjunto de todas as possíveis combinações lineares entre eles.
\end{lem}

\begin{proof}
Como $V'$ é um subespaço, ele é fechado na operação, com isso temos que $\alpha v_i \in V'$ para todo $\alpha \in \KK$, e como se $v,u \in V'$ então $v+u \in V'$ temos que os vetores da forma $\alpha_1v_1 + \alpha_2v_2 +\dots + \alpha_nv_n$ tem que estar em $V'$, como isso configura um espaço pois satisfaz todas as propriedades, como por exemplo ter 0 e ter inverso, e como todo subespaço que contém $v_1,v_2, \dots, v_n$ contém as combinações lineares deles, $V'$ é necessariamente o menor.
\end{proof}

Vale observar que se eu adicionar ao conjunto de vetores $v_1,v_2, \dots, v_n$ algum vetor $u$ tal que $u$ é uma combinação linear entre os $v$'s temos que o espaço gerado por $v_1,v_2, \dots, v_n$ e $v_1,v_2, \dots, v_n, u$ é o mesmo pois adicionamos alguém que já estava lá. Com isso temos:

\begin{defn}
Chamamos de base de $V$ qualquer conjuntos de vetores LI de $V$ tal que o espaço gerado por esse conjunto é $V$.
\end{defn}

\begin{lem}
\label{Lema1.3}
Se $v_1,v_2,\dots,v_n$ é uma base de $V$ então $\alpha_1v_1 + \alpha_2v_2 +\dots + \alpha_nv_n, v_2,\dots,v_n $ também é uma base de $V$ se $\alpha_1 \ne 0$
\end{lem}

\begin{proof}
Primeiro vamos mostrar que $\alpha_1v_1 + \alpha_2v_2 +\dots + \alpha_nv_n, v_2,\dots,v_n $ são linearmente independentes:

Suponha que existem $\beta_1,\beta_2,\cdots,\beta_n \in \KK$ tal que $\beta_1 \cdot (\alpha_1v_1 + \alpha_2v_2 +\dots + \alpha_n v_n) + \beta_2 \cdot v_2 + \dots + \beta_n \cdot  v_n = 0$, ou seja $ (\beta_1 \cdot \alpha_1) \cdot v_1 + (\beta_1 \cdot \alpha_2 + \beta_2)\cdot v_2 + \cdots (\beta_1 \cdot \alpha_n + \beta_n)\cdot v_n = 0$, como $v_1,v_2,\dots,v_n$ são linearmente independentes temos que $\beta_1 \cdot \alpha_1 = \beta_1 \cdot \alpha_2 + \beta_2 = \cdots = \beta_1 \cdot \alpha_n + \beta_n = 0$, como $\alpha_1 \ne 0$ temos que $\beta_1 = 0$ o que implica $\beta_2 = \beta_3 = \cdots = \beta_n = 0$, e assim temos que  $\alpha_1v_1 + \alpha_2v_2 +\dots + \alpha_nv_n, v_2,\dots,v_n$ são linearmente independentes.

Agora vamos mostrar que esses vetores $\alpha_1v_1 + \alpha_2v_2 +\dots + \alpha_nv_n, v_2,\dots,v_n $ geram o espaço inteiro:

Queremos mostrar que para todo vetor $u$  em $V$, $u$ é gerado pelos vetores, ou seja, existe $\gamma_1,\gamma_2, \dots, \gamma_n$ elementos do corpo
$\KK$, tal que $u = \gamma_1  (\alpha_1v_1 + \alpha_2v_2 +\dots + \alpha_nv_n) + \gamma_2 v_2 + \cdots + \gamma_n v_n$, arrumando a equação temos $u = (\gamma_1 \cdot \alpha_1) v_1 + (\gamma_1 \cdot \alpha_2 + \gamma_2) v_2 + (\gamma_1 \cdot \alpha_3 + \gamma_3) v_3 + \cdots + (\gamma_1 \cdot \alpha_n + \gamma_n) v_n$, contudo como $v_1,v_2,\cdots,v_n$ é uma base temos que existe $\beta_1,\beta_2,\cdots,\beta_n \in \KK$ tal que $\beta_1 v_1 + \beta_2 v_2 + \cdots + \beta_n v_n = u$, com isso temos que
$\gamma_1 = \beta_1 \cdot \alpha_1^{-1}$ e $\gamma_i = \beta_i - \beta_1 \cdot \alpha_1^{-1} \cdot \alpha_i$ para $i>1$.

Como $\alpha_1v_1 + \alpha_2v_2 +\dots + \alpha_nv_n, v_2,\dots,v_n$ geram o espaço todo e além disso são linearmente independentes, temos que eles formam uma base.
\end{proof} 

\newpage

\begin{thm}
Dado um espaço vetorial $V$ sobre $\KK$, todas as suas bases possuem o mesmo número de vetores.
\end{thm}


\begin{proof}
Suponha que existem duas bases com tamanhos distintos, seja $v_1,\allowbreak v_2,\allowbreak \dots, \allowbreak v_n$ a menor base de $V$ e $u_1,u_2,\dots,u_m$ outra base de $V$, com isso $m>n$.
Como $v_1 \in \KK$, temos que existe $\alpha_1,\alpha_2\dots,\alpha_m \in \KK$ tal que $v_1 = \alpha_1 u_1 + \alpha_2 u_2 + \cdots + \alpha_n u_n$.
Como $v_1$ é não nulo, temos que para algum $i$, $\alpha_i$ é diferente de 0, sem perda de generalidade suponha $\alpha_1 \ne 0$
 pelo lema \ref{Lema1.3}, temos que  $\alpha_1 u_1 + \alpha_2 u_2 + \cdots + \alpha_n u_n,u_2,\dots,u_m$ é uma base, ou seja, $v_1,u_2,\dots,u_m$ é uma base.
 
Como $v_2 \in V$ temos que existem $\beta_1,\beta_2,\dots,\beta_n \in \KK$ tal que $\beta_1 v_1 + \beta_2 u_2 + \cdots + \beta_n v_n = v_2$.
Se $\beta_2, \beta_3, \dots, \beta_n$ forem todos nulos, então temos que $v_1, v_2, \cdots, v_n$ são linearmente dependentes. Isso é um absurdo pois eles formariam uma base, logo existe $i>1$ com $\beta_i \neq 0$.
Suponha, sem perda de generalidade, que $\beta_2 \ne 0$. Pelo lema \ref{Lema1.3}, temos que $v_1,\beta_1 v_1 + \beta_2 u_2 + \cdots + \beta_n v_n,u_3\dots,u_m$ é uma base, ou seja $v_1,v_2,u_3\dots,u_m$ é uma base.

Podemos continuar esse processo indutivamente assim obtemos que $v_1, \allowbreak v_2, \allowbreak \dots, \allowbreak v_n, \allowbreak u_{n+1}, \allowbreak \dots, \allowbreak u_m$ é uma base,
contudo como $v_1,v_2,\dots,v_n$ é uma base, temos que eles geram $u_m$, com isso $v_1,v_2,\dots,v_n,u_{n+1},\dots,u_m$ são linearmente dependentes. Com isso temos um absurdo pois eles formam uma base, logo temos que as bases não podem ter tamanhos diferentes.
\end{proof}
