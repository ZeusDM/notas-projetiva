\chapter{Cônicas e Polaridades}
\section{Cônicas degeneradas}
\begin{defn}
Uma cônica é chamada degenerada se o seu polinômio pode ser fatorado em duas retas distintas.
\end{defn}

\begin{thm}
Uma cônica é degenerada se, e somente se, o determinante da matriz relacionada à polaridade é zero.
\end{thm}
\begin{proof}
Seja $M = \begin{pmatrix} a& d& e \\ d& b& f \\ e& f& c\end{pmatrix}$ a matriz da polaridade da cônica.

Vamos mostrar que se o polinômio é fatorável então o determinante é zero.
Suponha que $P(x,y,z) = ax^2+by^2+cy^2+2dxy+2exz+2fyz = (mx+ny+lz)\cdot(m'x+n'y+l'z)$. Com isso,

\begin{align*}
    a &= mm' \\
    b &= nn' \\
    c &= ll'\\
    2d &= nm'+n'm\\
    2e &= lm'+l'm\\
    2f &=ln'+l'n.
\end{align*}
Como $\det(M) = abc + 2def -e^2b-d^2c-f^2a$, substituindo a parametrização vemos que $\det(M) = 0$.

Agora vamos mostrar que se o determinante é zero então é fatorável.

Para isso vamos abrir em casos:
\begin{enumerate}
    \item $a=b=c=0$.
    \item Algum deles diferente de 0, sem perda de generalidade o $a$.
\end{enumerate}

Caso $a=b=c=0$, temos que $\det(M) = 2def = 0$. Sem perda de generalidade, $d=0$, ou seja:
\[ P(x,y,z) = 2exz+2fyz = 2\cdot(x+0\cdot y + 0 \cdot z)(0\cdot x + y + z).\]
Logo, $P(x,y,z)$ é fatorável.

Caso $a \ne 0$.
Considerando o $P(x,y,z)$ uma quadrática em $x$, temos que ele pode ser escrito como

\[ \left(x - \frac{(-dy - ez + \sqrt{\Delta_x})}{a}\right)\left(x - \frac{(-dy - ez - \sqrt{\Delta_x})}{a}\right),\]
com 

\[\Delta_x =-aby^2 - acz^2 - 2afyz + d^2y^2 + 2deyz + e^2z^2.\]

Se o $\Delta_x$ for quadrado perfeito como um polinômio teremos que é possível escrever $P(x,y,z)$ como produto de dois polinômios de grau $1$. Agora vamos dividir em mais dois casos:

\begin{enumerate}
    \item $d^2 = ab$ e $e^2 = ac$.
    \item Algum dentre $d^2 - ab$ e $e^2 - ac$ é não nulo. 
\end{enumerate}
Se $d^2 = ab$ e $e^2 = ac$, 

\[0 = \det M = -f^2a + 2def - abc\]
\[\implies f = \dfrac{2de\pm \sqrt{4d^2e^2-4a^2bc}}{2a}\]
\[\implies f = \dfrac{de}{a}\]
\[\implies af = de\]
\[\therefore \Delta_x = 0.\]

Agora, sem perda de generalidade, suponha $d^2\ne ab$. Então, $\Delta_x$ é uma quadrática em $y$, sendo o $\Delta$ dessa quadrática igual a 
\[-a^2bc + a^2f^2 + abe^2 + acd^2 - 2adef = -a\cdot\det(M) = 0,\]
logo o $\Delta_x $ é um trinômio do quadrado perfeito em $y$, portanto $P(x,y,z)$ é fatorável.
\end{proof}

\begin{thm}
Uma cônica é degenerada se, somente se, existem três pontos colineares pertencentes a ela.
\end{thm}

\begin{proof}
Se a cônica é degenerada ela é fatorável, logo existe toda uma reta contida nela, particularmente existem três pontos colineares.

Caso exista três pontos colineares:

Seja $A$,$B$ e $C$ os três pontos colineares contidos na cônica, seja $a$ e $b$ vetores representantes dos pontos $A$ e $B$ respectivamente, com isso temos que $C = [m\cdot a + n \cdot b]$ com $ m,n \ne 0$.

Primeiro, vamos olhar para cônica por sua matriz $M$, sabemos que se $A$, $B$ pertencem a cônica então \(a^T M a = 0 \text{ e } b^T M b = 0\), já como $C$ pertence a cônica temos:

\begin{align}
    (m\cdot a + n \cdot b)^T M (m\cdot a + n \cdot b) &= 0 \iff\\
    m^2 a^T M a + mn a^T M b + mn b^T M a + n^2 b^T M b &= 0 \iff \\
    a^T M b + b^T M a &= 0
\end{align}

Ou seja, temos que dado um par fixo $m,n$ nós conseguimos mostrar que $ a^T M b + b^T M a = 0$, fazendo o caminho de volta temos que para qualquer $x = \alpha a + \beta b$ com $\alpha,\beta \ne 0$, $x^T M x = 0$, ou seja $X$ pertence a cônica, como $A$ e $B$ já pertencem a cônica provamos que se há três pontos colineares numa cônica todos os pontos desta reta também estão na cônica.

Note que, se $a = (x_a,y_a,z_a)$, $b = (x_b,y_b,z_b)$ e $x_a/x_b = y_a/y_b = z_a/z_b$, $a$ e $b$ representariam o mesmo ponto projetivo logo, sem perda de generalidade, $y_a/y_b \ne z_a/z_b$.

Seja polinômio $C$ dessa cônica, e $R$ o da reta que contém $A$ e $B$, fazendo divisão polinomial na variável $x$, como $R$ tem grau 1 em $x$ temos que:
\[C(x,y,z) = R(x,y,z)\cdot Q(x,y,z) + P(y,z) \]
Sendo $P$ um polinômio de grau dois em $y$ e $z$, contudo para qualquer ponto tal que $R(x,y,z) = 0$ temos $C(x,y,z) = 0$ logo $P(y,z) = 0$, contudo como as raízes de $R$ são da forma $\alpha a + \beta b$ e $y_a/y_b \ne z_a/z_b$, temos que as combinação lineares de $a$ e $b$ cobrem todas as escolhas possíveis de $y$ e $z$, logo temos que para todos $y$ e $z$ pertencentes aos reais temos $P(y,z) = 0$, com isso $P \equiv 0$, ou seja $C(x,y,z) = R(x,y,z)\cdot Q(x,y,z)$, logo a cônica é degenerada.
\end{proof}