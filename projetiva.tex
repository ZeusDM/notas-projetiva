\documentclass{scrbook}
\KOMAoptions{
	fontsize = 11pt,
	parskip = half,
	paper = a4,
	twoside = false,
}
\usepackage[brazilian]{babel}	% Traduz as coisas automáticas (como datas, capítulos, etc) para PT-BR.
\usepackage{csquotes}
\usepackage{lmodern}
\usepackage{enumitem}			% Cria listas maneiras.
\usepackage{amsmath, amssymb, amsfonts}	% Packages da American Mathematical Society.
\usepackage[dvipsnames]{xcolor}
\usepackage[colorlinks]{hyperref}
\hypersetup{
	linkcolor=NavyBlue,
	citecolor=NavyBlue,
	filecolor=Red,
	urlcolor=NavyBlue,
	menucolor=Red,
	runcolor=Red}
\usepackage{graphicx} 
\usepackage{biblatex}
\addbibresource{referencias.bib}

% Definição de ambientes de teoremas.
\usepackage{amsthm}
\newtheorem{thm}{Teorema}[chapter]
\newtheorem{prop}[thm]{Proposição}
\newtheorem{lem}[thm]{Lema}
\newtheorem{cor}[thm]{Corolário}
\newtheorem{exer}[thm]{Exercício}
\newtheorem*{que}{Questão}
\theoremstyle{definition}
\newtheorem{defn}[thm]{Definição}
\newtheorem{pdefn}[thm]{Proposta de Definição}
\newtheorem{exmp}{Exemplo}[chapter]
\theoremstyle{remark}
\newtheorem*{sol}{Solução}

\newcommand{\RR}{\mathbb{R} }
\newcommand{\KK}{\mathbb{K} }
\newcommand{\ZZ}{\mathbb{Z} }
\newcommand{\PP}{\mathbb{P} }

\newcommand{\pvector}[3]{
    \begin{pmatrix}
        #1 \\
        #2 \\
        #3
    \end{pmatrix}
}

\newcommand{\bvector}[3]{
    \begin{bmatrix}
        #1 \\
        #2 \\
        #3
    \end{bmatrix}
}

\title{Geometria no Plano Projetivo}
\author{}
\date{Versão atualizada em \today}

% O seguinte comando diz quais "\include"s levar a sério, e quais desconsiderar.
% É útil mexer nisso para o TeX não precisar compilar o arquivo completo, caso você esteja mexendo em somente um capítulo.
%\includeonly{}

\begin{document}
	\maketitle

	\tableofcontents

	Esse pseudolivro é apenas um resumo, há grandes chances de existir erros nos provas e interpretações.

O foco desse livro é explorar as aplicações de geometria projetiva sobre espaços reais para resolver problemas de olimpíada, para isso é necessário que o leitor tenha alguma familiaridade com álgebra linear.
Uma coisa mais bonitinha.

	\part{Abordagem sem Coordenadas}

	%\chapter{A reta real não tem harmonia}

\section{Divisão Harmônica}

\begin{exer}
  Dados dois pontos \(A\) e \(B\) distintos na reta real, marque, com régua e compasso, todos os pontos \(P\) na reta real tais que \(\frac{AP}{BP} = 2\).
\end{exer}

Você observará que existem dois pontos, digamos \(P\) e \(Q\) --- um dentro do segmento \(AB\) e o outro fora do segmento \(AB\) ---  tais que \(\frac{AP}{BP} = \frac{AQ}{BQ} = 2\).

Por terem essa propriedade, dizemos que \(P\) e \(Q\) dividem \(A\) e \(B\) harmonicamente, na razão \(2\).

% Insira a figura A---P-------B-----------Q.

Não há nada especial com o número \(2\); \emph{parece} que, dado qualquer \(\lambda\) real positivo, existem sempre dois pontos que dividem \(A\) e \(B\) harmonicamente na razão \(\lambda\)...
Na verdade, tem um valor de \(\lambda\) para o qual isso não é verdade.

\begin{exer}
  Dados dois pontos \(A\) e \(B\) distintos na reta real, marque, com régua e compasso, todos os pontos \(P\) na reta real tais que \(\frac{AP}{BP} = 1\).
\end{exer}

Você observará que somente um ponto possui a propriedade acima; o ponto médio do segmento \(AB\).
Não existe ponto fora do segmento \(AB\) com essa propriedade --- de fato, um dos segmentos \(AP\) ou \(BP\) estará contido no outro, e portanto um será menor que o outro.


	%\chapter{Definindo o Plano Projetivo}

\section{Plano Euclidiano}

Você talvez esteja acostumado com a noção do Plano Euclidiano.
Essa estrutura involve muitos conceitos, incluindo (mas não limitado a):
\begin{enumerate}[label = \textbullet]
	\item pontos,
	\item retas,
	\item segmentos de reta,
	\item distância,
	\item ângulos,
	\item polígonos (triângulos, quadriláteros, \dots),
	\item circunferências,
	%\item conics (ellipses, parabolas, and hyperbolas),
	\item transformações (rotações, translações, reflexões, \dots).
\end{enumerate}

Apesar de todas esses conceitos estarem empacotadas no Plano Euclidiano, parece que os conceitos mais fundamentais são os primeiros dessa lista.
Em outras palavras, os conceitos de \emph{pontos} e \emph{retas} parecem ser mais simples e mais relacionados à definição de Plano Euclidiano, do que os conceitos de \emph{ângulos} e \emph{transformações}.

Nós poderíamos gastar algumas folhas formalizando a definição de Plano Euclidiano (usando o que os matemáticos chamam de \emph{axiomas}) e nos converncer de que, de fato, \emph{pontos} e \emph{retas} são intimamente relacionados ao que realmente o Plano Euclidiano é.
Porém, nós prefirimos não fazer isso.
Na verdade, vamos listar duas propriedades que o Plano Euclidiano \emph{parece} ter:
\begin{enumerate}[label = (\textit{\roman*})]
	\item Para cada par de pontos distindos, existe uma única reta passando por ambos. \label{item:pe:i}
	\item Para cada par de retas distintas, existe um único ponto que está em ambas. \label{item:pe:ii}
\end{enumerate}

\begin{exer} \label{exer:propriedadesdoplanoeuclidiano}
	Descubra se as propriedades acima são verdades ou não no Plano Euclideano.
\end{exer}

\section{Plano Projetivo}

No Exercício \ref{exer:propriedadesdoplanoeuclidiano}, você talvez tenha notado que, embora a propriedade \ref{item:pe:i} seja verdadeira, \ref{item:pe:ii} não é.
Existe até um nome especial para um par de retas que não se encontra em nenhum ponto: dizemos que essas retas são \emph{retas paralelas}.

% Aside: We could here define a relation between pair of lines, namely "line \(\ell_1\) is parallel to line \(\ell_2\)",
% and deduce that this relation is an equivalence relation; thus generates equivalence classes.

%In fact, one of the axioms of the Euclidean Plane directly contradicts \ref{item:pe:ii}:
%\begin{quote} % www.math.brown.edu/tbanchof/Beyond3d/chapter9/section01.html 
%	For every given point not on a given line, there exists a unique line passing through the given point that does not meet the given line.
%\end{quote}

Mas... as propriedades \ref{item:pe:i} e \ref{item:pe:ii} são tão bonitas (pelo menos pra nós).
É tão triste que elas não são verdadeiras.
Bem, nós somos matemáticos!
Nós podemos fazer o que quisermos com nossas definições (apesar de ter que viver com suas consequências).

\begin{pdefn}
	Um \emph{plano projetivo} é uma estrutura que consiste em três coisas:
	\begin{enumerate}[label = \textbullet]
		\item um conjunto \(\mathcal P\) de pontos,
		\item um conjunto \(\mathcal L\) de retas, e
		\item uma noção de incidência, i.e., se \(P\) é um ponto e \(\ell\) é uma reta, nós poderemos dizer se \(\ell\) passa (ou não) pelo ponto \(P\).
	\end{enumerate}

	Adicionalmente, precisamos das seguintes propriedades:
	\begin{enumerate}[label = (\textit{\roman*})]
		\item Para cada par de pontos distindos, existe uma única reta passando por ambos. \label{item:pp:i}
		\item Para cada par de retas distintas, existe um único ponto que está em ambas. \label{item:pp:ii}
	\end{enumerate}
\end{pdefn}


	\part{Abordagem com Coordenadas}

	
\chapter{Álgebra Linear}

\section{Algumas definições}

Antes de partimos para o tópico principal é necessário que algumas definições estejam claras.

%\subsection{Função}
\begin{defn}


$f$ é uma função do conjunto A no conjunto B (denota-se $f: \text{A} \to \text{B}$) se:
%\[ f = \{(a,b) \in A \times B | \ \forall a \in A \ \exists ! \ (a,b) \in f \}\]

\[f \subset A \times B \text{ tal que } \forall a \in A \ \exists ! \ b \in B \ | \ (a,b) \in f \]

Denotamos $f(a) = b \iff (a,b) \in f$
\end{defn}


\begin{defn}

%Uma operação binária \emph{estrela} em um conjunto X é, por definição, uma função de X $\times$ X em X, ou seja se denotarmos $x$ operação $x'$ como $x \star x'$ temos:

%\[ x \star x' =  y \iff f(x, x') = y\]


Uma operação binária \emph{"estrela"} em um conjunto A é, por definição, uma função de $A$ cartesiano $A$ em $A$, ou em matematiques:

\[ \star : A \times A \to A \]

Denotamos $a \star b = c \iff \star(a,b) = c$

Vale uma observação que pela nossa definição de função seria justo que estivesse escrito $\star((a,b))$, já que o elemento no qual aplicamos a função é o $(a,b)$, mas por uma certa comodidade ou talvez preguiça omitimos um dos parenteses. 

\end{defn}

\begin{defn}

Um corpo $\KK$ é um conjunto e duas operações, usualmente chamadas de adição e multiplicação, ou seja $\KK = \{C, +, \cdot\}$. Essas operações tem que ter algumas propriedades, sendo elas, para $a,b,c \in C$:

\begin{enumerate}
    \item Associativa:
    
        $a + (b + c) = (a + b) + c$
        $a \cdot (b \cdot c) = (a \cdot b) \cdot c = a \cdot b \cdot c$
        
    \item Comutativa:
    
    $a + b = b + a$
    
    $a \cdot b = b \cdot a$
    
    \item Elemento neutro:
    
    $\exists  e \in C \ | \ \forall a \in C, \ a + e = e + a = a$
    
    $\exists  e' \in C \ | \ \forall a \in C, \ a \cdot e' = e' \cdot a = a$
 
    \item Inverso:
    
   $\forall a \in C \ \exists \ a' \in C \ | \ a + a' = a' + a = e$
 
    $\forall a \ne e \in C \ \exists \ a' \in C \ | \ a \cdot a' = a' \cdot a= e'$
 
    \item Distributiva 
    
    $(a + b) \cdot c = a \cdot c + a \cdot b$
\end{enumerate}

É usual chamar o $e$ de $0$, o \textbf{$e'$} de $1$, o inverso aditivo de $a$ de $-a$ e o multiplicativo de $a^{-1}$ mesmo quando não estamos falando dos números reais. Além disso também é usual denotarmos $ a \in \KK $ ao invés de  $a \in C$.

\end{defn}

Dadas essa definições são bons exercícios mostrar que, para $a,b,c \in \KK $:
\begin{enumerate}
    \item $a \cdot (-1) = -a$
    \item $a \cdot 0 = 0$
\end{enumerate}



\section{Até que enfim Espaço}

\begin{defn}

Um espaço vetorial $V$ sobre o corpo $\KK$ é um conjunto munido de duas operações, $\oplus$ e $\odot$. Sendo $\oplus: V \times V \to V \text{ e } \odot: \KK \times V \to V$

%Um espaço vetorial é um conjunto, um corpo e duas operações, uma do espaço no espaço e uma operação do corpo no espaço, ou seja:


Sendo que estas possuem algumas propriedades, para $v,u,w \in V \text{ e } \alpha, \beta \in \KK$, temos:

\begin{enumerate}
    \item Associativa em $\oplus$ e $\odot$
    
        $v \oplus (u \oplus w) = (v \oplus u) \oplus w $
        
        $\alpha \odot (\beta \odot v) = \beta \odot (\alpha \odot v) = (\alpha \cdot \beta)\odot v$
    \item Comutativa em $\oplus$
    
        $v \oplus u = u \oplus v$
    \item Elemento neutro
    
     $\exists e \in V \ | \  \forall v \in V, v \oplus e = e \oplus v = v$, usualmente denotamos esse elemento como 0.
     
     $1 \odot v = v$, sendo 1 o elemento neutro da multiplicação em $\KK$.
     
     \item Inverso
     
     $\forall v \in V , \exists v' \in V \ | \ v \oplus v' = e$ usualmente denotamos $v'$ por $-v$ 
    \item Distributiva
    
    $(\alpha + \beta)\odot v = \alpha \odot v \oplus \beta \odot v$
    
    $\alpha \odot (v \oplus u)= \alpha \odot v + \alpha \odot u$
\end{enumerate}
\end{defn}

Vale observar, que como em corpos, também é usual simplificarmos um pouco a notação, usando o mesmo símbolo para soma (+) e o produto por escalar é usualmente denotado apenas pela justaposição dos elementos do corpo e do espaço ($\alpha v$), ainda no produto vale ressaltar que só está definido $\alpha v$, logo sempre que estiver escrito $v \alpha$, queremos dizer $\alpha v$.

Em resumo, tudo que der pra escrever menos, nós vamos escrever menos.

São bons exercícios mostrar que:

\begin{enumerate}
    \item $0v=0$
    \item $(-1)v = -v$
\end{enumerate}

\begin{defn}

Chamamos o espaço vetorial $V'$ de um subespaço de $V$ se: 
\begin{enumerate}
    \item $V'$ esta definido sobre o mesmo corpo.
    \item $\forall v \in V', v \in V$
    \item As operações de $V'$ são as operações de $V$ com o domínio e contradomínio restritos à $V'$. 
    
%    \item $\forall (v,u) \in V' \times V'$
%    
%    $v+u \in V'$
%    
%    $ -v \in V'$
%    \item $0 \in V'$
\end{enumerate}
\end{defn}

\subsection{Coisas relacionadas com base}

As  definições a seguir podem parecer um pouco desconexas separadas, contudo após elas todas estarem claras, o motivo da existência de cada uma fica mais claro também.


\begin{defn}
Falamos que o vetor $u$  é uma combinação linear entre os vetores $v_1,v_2,\dots,v_n$, quando quando existe $\alpha_1, \alpha_2, \cdots , \alpha_n \in \KK$ tal que $u = \alpha_1v_1 + \alpha_2v_2 +\dots + \alpha_nv_n $.
\end{defn}

\begin{defn}
Chamamos os vetores $v_1,v_2, \dots, v_n$ de linearmente independentes quando nenhum deles podem ser escritos como uma combinação linear dos outros vetores.
\end{defn}

\begin{cor}
$v_1,v_2, \dots, v_n$ são LI (linaermente independes) $\iff \alpha_1v_1 + \alpha_2v_2 +\dots + \alpha_nv_n = 0 \implies \alpha_1 = \alpha_2 = \dots = \alpha_n =0$.
\end{cor}

\begin{defn}
Um espaço $V'$ gerado pelos vetores $v_1,v_2,\dots,v_n$ de $V$ é o menor subespaço de $V$ que contém $v_1,v_2,\dots , v_n$.

Obs.: "Menor" significa que não existe nenhum subespaço que contenha esses vetores e  esteja contido em $V'$, tirando o próprio $V'$.
\end{defn}
    
\begin{lem}
O espaço gerados pelos vetores $v_1,v_2, \dots, v_n$ é o conjunto de todas as possíveis combinações lineares entre eles.
\end{lem}

\begin{proof}
Como $V'$ é um subespaço, ele é fechado na operação, com isso temos que $\alpha v_i \in V'$ para todo $\alpha \in \KK$, e como se $v,u \in V'$ então $v+u \in V'$ temos que os vetores da forma $\alpha_1v_1 + \alpha_2v_2 +\dots + \alpha_nv_n$ tem que estar em $V'$, como isso configura um espaço pois satisfaz todas as propriedades, como por exemplo ter 0 e ter inverso, e como todo subespaço que contém $v_1,v_2, \dots, v_n$ contém as combinações lineares deles, $V'$ é necessariamente o menor.
\end{proof}

Vale observar que se eu adicionar ao conjunto de vetores $v_1,v_2, \dots, v_n$ algum vetor $u$ tal que $u$ é uma combinação linear entre os $v$'s temos que o espaço gerado por $v_1,v_2, \dots, v_n$ e $v_1,v_2, \dots, v_n, u$ é o mesmo pois adicionamos alguém que já estava lá. Com isso temos:

\begin{defn}
Chamamos de base de $V$ qualquer conjuntos de vetores LI de $V$ tal que o espaço gerado por esse conjunto é $V$.
\end{defn}

\begin{lem}
\label{Lema1.3}
Se $v_1,v_2,\dots,v_n$ é uma base de $V$ então $\alpha_1v_1 + \alpha_2v_2 +\dots + \alpha_nv_n, v_2,\dots,v_n $ também é uma base de $V$ se $\alpha_1 \ne 0$
\end{lem}

\begin{proof}
Primeiro vamos mostrar que $\alpha_1v_1 + \alpha_2v_2 +\dots + \alpha_nv_n, v_2,\dots,v_n $ são linearmente independentes:

Suponha que existem $\beta_1,\beta_2,\cdots,\beta_n \in \KK$ tal que $\beta_1 \cdot (\alpha_1v_1 + \alpha_2v_2 +\dots + \alpha_n v_n) + \beta_2 \cdot v_2 + \dots + \beta_n \cdot  v_n = 0$, ou seja $ (\beta_1 \cdot \alpha_1) \cdot v_1 + (\beta_1 \cdot \alpha_2 + \beta_2)\cdot v_2 + \cdots (\beta_1 \cdot \alpha_n + \beta_n)\cdot v_n = 0$, como $v_1,v_2,\dots,v_n$ são linearmente independentes temos que $\beta_1 \cdot \alpha_1 = \beta_1 \cdot \alpha_2 + \beta_2 = \cdots = \beta_1 \cdot \alpha_n + \beta_n = 0$, como $\alpha_1 \ne 0$ temos que $\beta_1 = 0$ o que implica $\beta_2 = \beta_3 = \cdots = \beta_n = 0$, e assim temos que  $\alpha_1v_1 + \alpha_2v_2 +\dots + \alpha_nv_n, v_2,\dots,v_n$ são linearmente independentes.

Agora vamos mostrar que esses vetores $\alpha_1v_1 + \alpha_2v_2 +\dots + \alpha_nv_n, v_2,\dots,v_n $ geram o espaço inteiro:

Queremos mostrar que para todo vetor $u$  em $V$, $u$ é gerado pelos vetores, ou seja, existe $\gamma_1,\gamma_2, \dots, \gamma_n$ elementos do corpo
$\KK$, tal que $u = \gamma_1  (\alpha_1v_1 + \alpha_2v_2 +\dots + \alpha_nv_n) + \gamma_2 v_2 + \cdots + \gamma_n v_n$, arrumando a equação temos $u = (\gamma_1 \cdot \alpha_1) v_1 + (\gamma_1 \cdot \alpha_2 + \gamma_2) v_2 + (\gamma_1 \cdot \alpha_3 + \gamma_3) v_3 + \cdots + (\gamma_1 \cdot \alpha_n + \gamma_n) v_n$, contudo como $v_1,v_2,\cdots,v_n$ é uma base temos que existe $\beta_1,\beta_2,\cdots,\beta_n \in \KK$ tal que $\beta_1 v_1 + \beta_2 v_2 + \cdots + \beta_n v_n = u$, com isso temos que
$\gamma_1 = \beta_1 \cdot \alpha_1^{-1}$ e $\gamma_i = \beta_i - \beta_1 \cdot \alpha_1^{-1} \cdot \alpha_i$ para $i>1$.

Como $\alpha_1v_1 + \alpha_2v_2 +\dots + \alpha_nv_n, v_2,\dots,v_n$ geram o espaço todo e além disso são linearmente independentes, temos que eles formam uma base.
\end{proof} 

\newpage

\begin{thm}
Dado um espaço vetorial $V$ sobre $\KK$, todas as suas bases possuem o mesmo número de vetores.
\end{thm}


\begin{proof}
Suponha que existem duas bases com tamanhos distintos, seja $v_1,v_2,\dots,v_n$ a menor base de $V$ e $u_1,u_2,\dots,u_m$ outra base de $V$, com isso $m>n$.
Como $v_1 \in \KK$, temos que existe $\alpha_1,\alpha_2\dots,\alpha_m \in \KK$ tal que $v_1 = \alpha_1 u_1 + \alpha_2 u_2 + \cdots + \alpha_n u_n$.
Como $v_1$ é não nulo, temos que para algum $i$, $\alpha_i$ é diferente de 0, sem perda de generalidade suponha $\alpha_1 \ne 0$
 pelo lema \ref{Lema1.3}, temos que  $\alpha_1 u_1 + \alpha_2 u_2 + \cdots + \alpha_n u_n,u_2,\dots,u_m$ é uma base, ou seja, $v_1,u_2,\dots,u_m$ é uma base.
 
Como $v_2 \in V$ temos que existem $\beta_1,\beta_2,\dots,\beta_n \in \KK$ tal que $\beta_1 v_1 + \beta_2 u_2 + \cdots + \beta_n v_n = v_2$.
Se $\beta_2, \beta_3, \dots, \beta_n$ forem todos nulos, então temos que $v_1, v_2, \cdots, v_n$ são linearmente dependentes. Isso é um absurdo pois eles formariam uma base, logo existe $i>1$ com $\beta_i \neq 0$.
Suponha, sem perda de generalidade, que $\beta_2 \ne 0$. Pelo lema \ref{Lema1.3}, temos que $v_1,\beta_1 v_1 + \beta_2 u_2 + \cdots + \beta_n v_n,u_3\dots,u_m$ é uma base, ou seja $v_1,v_2,u_3\dots,u_m$ é uma base.

Podemos continuar esse processo indutivamente assim obtemos que $v_1,v_2,\dots,v_n,u_{n+1},\dots,u_m$ é uma base,
contudo como $v_1,v_2,\dots,v_n$ é uma base, temos que eles geram $u_m$, com isso $v_1,v_2,\dots,v_n,u_{n+1},\dots,u_m$ são linearmente dependentes. Com isso temos um absurdo pois eles formam uma base, logo temos que as bases não podem ter tamanhos diferentes.
\end{proof}
	\chapter{Reta Projetiva}

\section{Definição}

\begin{defn}[Reta Projetiva]
A \emph{reta projetiva real \(\mathbb{P}^1\mathbb{R}\)} é o conjunto das retas do \(\mathbb{R}^2\) que passam pela origem.
Elementos da reta projetiva real são chamados de \emph{pontos projetivos}.
\end{defn}

Como um abuso de notação, chamaremos a reta projetiva real somente de reta projetiva.
Quando o significado estiver claro, podemos chamar a reta projetiva de reta, e o ponto projetivo de ponto.

Em termos de Álgebra Linear, os pontos da reta projetiva são os subespaços do $\mathbb{R}^2$ com dimensão~$1$.

Os elementos de \(\mathbb{R}^2\) são pares ordenados \((a, b)\), com \(a, b\) reais.
Usando essa representação, podemos inferir uma representação dos pontos projetivos da reta projetiva.
Se \((a, b) \neq (0, 0)\), vamos chamar de \([a, b]\) a única reta que passa pela origem e por \((a, b)\).
Lembre-se que uma reta no \(\mathbb{R}^2\) é definida por dois pontos distintos; portanto, a reta \([a, b]\) está bem definida.

De modo mais preciso, podemos concluir que \([a, b]\) é o conjunto dos múltiplos reais de \((a, b)\), isto é, é o conjunto \(\{ (\lambda a, \lambda b) : \lambda \in \mathbb{R}\}\).

Porém, essa representação não é única.
Por exemplo, \([1, 2]\), \([2, 4]\) e \([-1, -2]\) representam o mesmo ponto projetivo, isto é, a mesma reta.
De modo geral, se \(\lambda \neq 0\), os pontos projetivos \([a, b]\) e \([\lambda a, \lambda b]\) são os mesmos.

Apesar da representação não ser única, temos que \([a, b] = [c, d]\) se, e somente se, \((c, d)\) é um múltiplo real de \((a, b)\).
Em outras palavras, \([a, b] = [c, d]\) se, e somente se, existe \(\lambda \in \mathbb{R}\) tal que \((c, d) = (\lambda a, \lambda b)\).

Portanto, o sistema de coordenadas da reta projetiva é um pouco diferente do sistema que estamos acostumados, mas continua sendo útil mesmo assim. Chamamos sistemas como esse, em que múltiplicar por um escalar não muda o ponto, de \emph{sistema de coordenadas homogêneas}.

\begin{exmp}
Os pontos projetivos $[1,0]$, $[0,1]$, $[1,1]$ e $[2,1]$ correspondem respectivamente ao eixo $x$, ao eixo $y$, à reta $y=x$ e à reta $y=\frac{1}{2}x$ do $\mathbb{R}^2$.
\end{exmp}

Mas por que chamamos a reta projetiva de ``reta''? Uma maneira de responder essa pergunta é, dada uma base do $\mathbb{R}^2$, considerar a reta $y=1$. Com isso, podemos tomar como representante para cada ponto projetivo o vetor que leva da origem até sua interseção  com a reta $y=1$.

(inserir imagem)

Contudo, vemos que exatamente um dos pontos projetivos, a reta \(y = 0\), não intersecta a reta \(y = 1\).
Para essa reta, vamos escolher o representante $[1,0]$.
Portanto, \[\mathbb{P}^1\mathbb{R} = \{ [x,1] : x \in \mathbb{R} \} \cup \{[1,0]\},\]
ou seja, podemos interpretar reta projetiva real como uma reta real com um ponto extra.

\section{Projetividades}

	\chapter{Plano Projetivo}

\section{Definição}

\begin{defn}[Plano Projetivo Real] \label{defn:planoprojetivoreal}
O \emph{plano projetivo real \(\mathbb{P}^2(\mathbb{R})\)} é o conjunto das retas do \(\mathbb{R}^3\) que passam pela origem.
Elementos do plano projetivo real são chamados de \emph{pontos projetivos}.
\end{defn}

Essa definição é completamente análoga à Definição \ref{defn:retaprojetivareal}; apenas aumentamos uma dimensão.

Como um abuso de notação, chamaremos o plano projetivo real somente de plano projetivo.
Quando o significado estiver claro, podemos chamar o plano projetivo de plano.

Em termos de Álgebra Linear, os pontos do plano projetivo são os subespaços do \(\mathbb{R}^3\) com dimensão~\(1\).

Podemos inferir uma representação dos pontos projetivos do plano projetivo a partir dos vetores do \(\mathbb{R}^3\), de modo completamente análogo ao que fizemos na Seção \ref{sec:defnretaprojetiva}.

Se \(\mathbf{v} \neq (0, 0, 0)\), vamos chamar de \([\mathbf{v}]\) a única reta que passa pela origem e por \(\mathbf{v}\); equivalentemente, também podemos definir \([\mathbf{v}]\) como \(\langle v \rangle\), o subespaço do \(\mathbb{R}^3\) gerado por \(\mathbf{v}\).
Podemos concluir que \([\mathbf{v}]\) é o conjunto dos múltiplos reais de \(\mathbf{v}\), isto é, é o conjunto \(\{ \lambda \mathbf{v} : \lambda \in \mathbb{R}\}\).

Portanto, temos novamente que \([\mathbf{v}] = [\mathbf{u}]\) se, e somente se, \(\mathbf{u}\) é um múltiplo real de \(\mathbf{v}\).
Em outras palavras, \([\mathbf{v}] = [\mathbf{u}]\) se, e somente se, existe \(\lambda \in \mathbb{R}\) tal que \(\mathbf{u} = \lambda\mathbf{v}\).

Por que chamamos o plano projetivo de ``plano''? Uma maneira de responder essa pergunta é, dada uma base do $\mathbb{R}^3$, considerar o plano $z=1$.
Com isso, podemos tomar como representante para cada ponto projetivo o vetor que leva da origem até sua interseção com o plano $z=1$.

Contudo, vemos que vários pontos projetivos não intersectam o plano \(z = 1\).
Mais precisamente, as retas que passam pela origem e estão contidas no plano \(z = 0\) são os pontos projetivos que não possuem representante pela regra acima.
Porém, o conjunto de pontos projetivos contidos em um plano real possui nome!
Chamamos esses conjuntos de reta projetiva.

Portanto, podemos interpretar a reta projetiva real como um plano real com uma reta projetiva extra.
Juntando com o que desenvolvemos na Seção \ref{sec:defnretaprojetiva}, podemos dizer que a reta projetiva real é um plano real, mais uma reta real, mais um ponto extra.
Podemos ver isso claramente em \ref{eqn:planoprojetivoplanoretaponto}.
\begin{equation} \label{eqn:planoprojetivoplanoretaponto}
  \mathbb{P}^2(\mathbb{R}) = \{ [x, y, 1] : x, y \in \mathbb{R} \} \cup \{[x, 1, 0] : x \in \mathbb{R}\} \cup \{[1, 0, 0]\}.
\end{equation}

\begin{exer}
  Convença a si mesmo de que a igualdade de conjuntos em \ref{eqn:planoprojetivoplanoretaponto} é verdadeira.
\end{exer}

	%\chapter{Projeções e Projetividades}

	%\chapter{Cônicas e Polaridades}


	\printbibliography

\end{document}
